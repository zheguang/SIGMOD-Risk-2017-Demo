\documentclass{sig}

\usepackage{algorithm}
\usepackage{algpseudocode}
\usepackage{amsmath}
\usepackage{appendix}
\usepackage{balance}
\usepackage{caption}
\usepackage{color}
\usepackage{comment}
\usepackage{epstopdf}
\usepackage{graphicx}
\usepackage{listings}
\usepackage{mathtools}
\usepackage{microtype}
\usepackage{multirow}
\usepackage{pdfpages}
\usepackage{subcaption}
\usepackage{subfig}
\usepackage{tabularx}
\usepackage{times}
\usepackage{url}
\usepackage{xcolor}
\usepackage{xspace}
\usepackage{bm}
\usepackage{mdwlist}

\newcommand{\subparagraph}{}
\usepackage{titlesec}

%\titlespacing{\section}{0ex}{0.8ex}{0ex}
%\titlespacing{\subsection}{0ex}{0.8ex}{0ex}
%\titlespacing{\subsubsection}{0ex}{0.7ex}{0ex}
%\setlength{\parskip}{0ex}

\algrenewcommand\alglinenumber[1]{\scriptsize #1:}

\makeatletter
\renewcommand*{\ALG@name}{Investing Rule}
\makeatother


\usepackage{breqn}
\interfootnotelinepenalty=10000

\newcommand{\todo}[1]{\textcolor{blue}{TODO: #1}}
\newcommand{\done}[1]{\textcolor{green}{DONE#1}}
\newcommand{\tim}[1]{\textcolor{red}{Tim: #1}}
\newcommand{\lore}[1]{\textcolor{blue}{Lorenzo: #1}}
\newcommand{\ez}[1]{\textcolor{orange}{ez: #1}}
\newcommand{\sam}[1]{\textcolor{purple}{sam: #1}}
\newcommand{\carsten}[1]{\textcolor{brown}{Carsten: #1}}

\newcommand{\naive}{na\"{\i}ve\xspace}
\newcommand{\Naive}{Na\"{\i}ve\xspace}
\newcommand{\naively}{na\"{\i}vely\xspace}

\newcommand{\ainv}{$\alpha$-investing }
\newcommand{\sfdr}{Sequential-FDR }
\newcommand{\mfdre}{$mFDR_{\eta}$ }
\newcommand{\mfdr}[1]{$mFDR_{#1}$}
\newcommand{\Ex}[1]{E\left[#1\right]}
\newcommand{\Prob}[1]{\text{P}\left(#1\right)}
\newcommand{\CProb}[2]{\text{P}\left(#1|#2\right)}
\newcommand{\pval}{$p$-value }
\newcommand{\pvals}{$p$-values }


\newcommand{\system}{{\sc Aware}}
\newcommand{\Chi}{\mathcal{X}}

\newtheorem{theorem}{Theorem}
\newtheorem{lemma}{Lemma}


%\setlength\abovedisplayskip{0pt plus 2pt minus 0pt}
%\setlength\belowdisplayskip{0pt plus 2pt minus 0pt}

%\makeatletter
%\g@addto@macro \normalsize {%
%\setlength\abovedisplayskip{0pt plus 2pt minus 0pt}%
%\setlength\belowdisplayskip{0pt plus 2pt minus 0pt}%
%\setlength\abovedisplayshortskip{0pt plus 2pt minus 0pt}%
%\setlength\belowdisplayshortskip{0pt plus 2pt minus 0pt}%
%}
%\makeatother

%\makeatletter
%\g@addto@macro \small {%
%\setlength\abovedisplayskip{0pt plus 2pt minus 0pt}%
%\setlength\belowdisplayskip{2pt plus 2pt minus 0pt}%
%\setlength\abovedisplayshortskip{0pt plus 2pt minus 0pt}%
%\setlength\belowdisplayshortskip{2pt plus 2pt minus 0pt}%
%}
%\makeatother

\DeclareMathOperator*{\argmin}{\arg \min}
\DeclarePairedDelimiter\ceil{\lceil}{\rceil}

\captionsetup{font={bf}}

%\newcommand{\system}{{\sc Vizdom}}

\newcommand{\specialcell}[2][c]{
  \begin{tabular}[#1]{@{}c@{}}#2\end{tabular}}

\definecolor{dark-gray}{gray}{0.2}

\newenvironment{packed_item}{
\begin{list}{$\bullet$}{
  \setlength{\itemsep}{-2pt}
  \setlength{\parskip}{1pt}
  \setlength{\labelwidth}{15 pt}
  \setlength{\leftmargin}{10pt}
  \setlength{\itemindent}{0pt}}
}{\end{list}}

% for floated 2 column equations
\newcounter{tempEquationCounter}
\newcounter{thisEquationNumber}
\newenvironment{floatEq}
{\setcounter{thisEquationNumber}{\value{equation}}\addtocounter{equation}{1}% record equation as happened and remember number
\begin{figure*}[t]% float following equation across columns
\normalsize\setcounter{tempEquationCounter}{\value{equation}}% record current equation number in floated location
\setcounter{equation}{\value{thisEquationNumber}}% use previous equation number
}
{\setcounter{equation}{\value{tempEquationCounter}}% set back to equation number in floated location
\hrulefill\vspace*{4pt}% add a horizontal rule separator
\end{figure*}% end float environment
}
\newtheorem{defi}{Definition}
\makeatletter
\newenvironment{subheuristic}[1]{%
  \def\subtheoremcounter{#1}%
  \refstepcounter{#1}%
  \protected@edef\theparentnumber{\csname the#1\endcsname}%
  \setcounter{parentnumber}{\value{#1}}%
  \setcounter{#1}{0}%
  \expandafter\def\csname the#1\endcsname{\theparentnumber\alph{#1}}%
  \ignorespaces
}{%
  \setcounter{\subtheoremcounter}{\value{parentnumber}}%
  \ignorespacesafterend
}
\makeatother
\newcounter{parentnumber}

\newtheorem{heuristic}{Heuristic}

\begin{document}



\title{Aware: Visual Data Exploration that Controls False Discoveries}

\numberofauthors{1}
\author{
\alignauthor
%\vspace*{-30pt}
%Paper \#178
\begin{tabular}{cccc}
\end{tabular}\\
%\vspace{1.5mm}
\affaddr{Department of Computer Science, Brown University}\\
%\vspace{0.75mm}
\{firstname\_lastname\}@brown.edu
%\email{\{firstname\_lastname\}@brown.edu}
}

%\numberofauthors{6}
%\author{
%\small{Andrew Crotty, Alex Galakatos, Kayhan Dursun, Tim Kraska, Ugur Cetintemel, Stan Zdonik} \\
%\small{Department of Computer Science, Brown University} \\
%\small{\{crottyan, agg, kayhan, kraskat, ugur, sbz\}@cs.brown.edu}
%}
\date{}
\maketitle

\begin{abstract}
Exploring data via visualizations has become a popular way to understand complex data. Observations on the visual results are inherently hypotheses.  However, visually significant information may not carry any statistical significance. Moreover, visual data exploration produces observations dynamically and results in increasing numbers of false discoveries.  We present our solution based on Vizdom~\cite{vizdom}, namely, \system{}, a visual data exploration system that interacts with user to formulate visualization-based hypotheses and provides interactive control of false discoveries based on the recent advance in the subject~\cite{controlling-false-discoveries}.
\end{abstract}


\section{Introduction}
\label{sec:intro}
 In the era of Big Data, visualization arises as an important mean to explore and derive insights from data.  However, visualized information such as relationships and trends may emerge from random noise.  Without proper statistical control, users may mistake visually significant observations as statistically significant.  Worse yet, systems that search and recommend visualized information would produce large number of questionable insights.  
A recent study shows that visualization and recommendation without considering the risk of false discovery, such as Vizdom~\cite{vizdom}, SeeDB~\cite{seedb} and Data Polygamy~\cite{polygamy}, becomes unusable on datasets with any randomness \cite{towards-sustainable-insight}.

\sam{insert figures} False discovery due to random noise is pervasive in visual data exploration on real-world datasets. For example, in a recently conducted survey on personal habits and opinions~\cite{towards-sustainable-insight}, the disbelief in alien existence combining with the preference of potato chips produces visually different proportion of workspace preferences, as visualized on Vizdom~\cite{vizdom}.  But such predictor proves not only conceptually questionable but also statistically insignificant in its correlation to the prediction.  More alarmingly, more and more such observation-based hypotheses may be formulated as the user continues to explore the dataset, and hence quickly increase the expected number of false discoveries. For an example of the same survey, after searching through a few different comparisons, we stumbled upon a visualization that suggests hair color predicts whether one knows about Michael Stonebraker, and it would be statistically significant if considered as a lone hypothesis. This phenomenon is often referred to as data dredging or $p$-hacking~\cite{p-hacking}, and formally known as the multiple comparison problem~\cite{shaffer1995multiple}.

Several challenges exist to control the false discoveries in interactive data exploration.  To begin with, most statisticians evaluate hypotheses in a passive computing environment such as R~\cite{R}. In visual data exploration, the system also needs to assist the user in the hypothesis formulation and the hypothesis test selection. Secondly, interactive data exploration generates hypotheses dynamically to follow the process of human decision making.  However the traditional statistical procedures such as Bonferroni~\cite{bonferroni1936teoria} and Sequential FDR~\cite{seq-fdr} are not dynamic in that they require knowing all the hypotheses a priori before finalizing the inference on statistical significance.  Moreover, these traditional techniques assume complete pass of the data, and hence would delay user interaction on larger datasets.  We believe that progressive computation as an appealing paradigm for Big Data visualization~\cite{emanuel-user-study, online-aggregation, vizdom}.  Thus, we implemented a recent advance on interactive false discovery control procedure in data exploration that is both dynamic and progressive~\cite{controlling-false-discoveries}.

In summary, we present \system{} as our solution based on Vizdom~\cite{vizdom} that addresses the aforementioned challenges, namely,
\begin{itemize}
    \item To formulate hypotheses via user interaction;
    \item To visualize the statistical significance and other contextual information for each observation;
    \item To control multiple hypotheses dynamically during data exploration;
    \item To progressively compute the risk of false discovery.
\end{itemize}


\section{System Design}
\label{sec:system-design}

\subsection{User Interface}
\label{sec:ui}

\begin{comment}
As argued in the previous section, user feedback is essential in determining, tracking and controlling the right hypothesis during the data exploration process.
With \system~we created a system that applies our heuristic automatically to all visualizations. We designed \system~'s user interface with a few goals in mind.

First, the user should be able to see the hypotheses the system assumed so far, their \pvals, effect sizes and if they are considered significant and should be able to change, add or delete hypotheses at any given stage of the exploration. 

Second, hypotheses rejection decisions should never change based on future user actions unless the user explicitly asks for it. We therefore require an incremental procedure to control the multiple hypothesis risk that does not change its rejection decisions even if more hypothesis tests are executed.
For example, the system should not state that their is a significant age difference for not married highly educated people, and then later on revoke its assessment just because the user did more tests. 
More formally, if the system determined which hypotheses $m_1 ...  m_n$ are significant (i.e., it rejects the null) or not and the user changes the last hypothesis or adds an hypothesis $m_{n+1}$, which should be the most common cases, the significance of hypotheses $m_1..m_{n}$ should not change. 
However, if the user might change, delete, or add hypothesis $k \in {1,..,n}$, depending on the used procedure we might allow that the significance of hypotheses $m_{k+1}$ to $m_n$ might have to change as well.
%Furthermore, the system automatically reports on the {\bf effect size} as a color coded value based on Cohen's suggestions \cite{something}.
%This is according to best practice, as the effect size (e.g., the age difference) determines how big the observed difference is compared to the variance. 

Third, individual hypothesis descriptions should be augmented with information about how much data $n^{H1}$ the user has to add, under the assumption that the new data will follow the current observed distribution of the data, to make an hypothesis significant. 
While sounding counter-intuitive, as one might (wrongly) imply, it is possible to make any hypothesis true by adding more data, calculating this value is in some fields already common practice. 
For example, in genetics scientist often search (automatically) for correlations between genes and high-level effects (like cancer). 
If such a correlation is found, often because of the multiple hypothesis error the chance of a true discovery is tiny (i.e., the \pval is too high). 
In that case the scientist works backwards and estimates how much more genes she has to to sequence in order to make the hypothesis relevant, expecting that the new data (e.g., gene sequences) follow the same distribution of the data the scientist already has.
However, if the effect was just produced by chance, the new data will be more similar to the distribution of the null-hypothesis and the null will not be rejected.  
%Similar, it is possible for a rejected null-hypothesis to calculate how much data $n^{H0}$ has to be added if the null-hypothesis is true, until the null-hypothesis will be accepted. 
% Eli: This has no statistical meaning.
The required value is generally easy to calculate or approximate,  and are highly valuable for the end-user. 
A small value for $n^{H1}$ in relation to the number of totally tested hypotheses might be an indication that the power (i.e., the chance to accept a true alternative hypothesis) of the test was not sufficiently large. 

And finally, users should be able to bookmark important hypotheses. 
Our system uses default hypothesis throughout the exploration and the user might find it too cumbersome to correct everyone for his real intentions, there might be more hypotheses generated than the user intended to test. 
Even if all hypotheses are what the user was considering, some of them might be more important to her than others; the hypotheses the user would like to include in a presentation or show to her boss. 
A key key question becomes, what is the expected number of false discoveries among those important discoveries?
\end{comment}

\begin{figure}
\centering
\includegraphics[width=0.48\textwidth]{figures/risk_controller}
\caption{The \system{} User Interface}
\label{fig:ui}	
\end{figure}

\sam{todo: how to let user specify hypotheses; remove (B); hypothesis re-test once data added; past hypothesis lookup/redraw}
The user interface of \system{} features an unbounded 2D canvas where visualizations can be laid out in a free form fashion (Figure~\ref{fig:ui}). A ``risk-gauge'' on the right-hand side of the display (Figure~\ref{fig:ui} (A)) serves two purposes: 
\begin{itemize}
    \item To give user a summary of the control procedure (e.g., the budget for the false discovery rate set to 5\% with current remaining wealth of 2.5\%;
    \item To provide access to a scrollable list of all the hypotheses that have been explored.
\end{itemize}
Each list entry displays details about an observation and its statistical significance.  The text labels describe the null and alternative hypotheses concerning each observation and the corresponding hypothesis tests and \pvals. Each color coded tile indicates whether the observation is statistically significant or insignificant, which corresponds to green or red respectively.  The distributions of null and alternative hypotheses and the color coded effect size are also visualized (D).  To help the user understand the effect of data collection, the sample size estimate for the current significance level is displayed for each hypothesis test assuming the effect size is fixed (C).  For example, the five green squares in (C) indicates approximately five times the current data size with the same effect size would make this observation significant. Finally, important observations can be marked by tapping the ``star'' icons (E).

\subsection{Backend}
\label{sec:backend}
\sam{todo: describe system design choices, such as how to generate null hypotheses given user interactions, how to progressively compute marginalized false discovery rate in parallel, etc.}

\section{Demo Proposal}
\label{sec:scenarios}

\begin{figure*}[h]
\center
\includegraphics[width=\textwidth]{figures/storyboard.pdf}
\caption{Storyboard showing a user exploring a dataset through \system{}.}
\label{fig:sb}
\end{figure*}

To demonstrate the features of \system{}, we will use various publicly available datasets as well as a dataset obtained through a survey on Amazon Mechanical Turk~\cite{binnig2017sustainable}. In this survey we collected answer to 69 questions from 104 participants. Questions cover wide range of habits and opinions and are mostly unrelated conceptually, such as ``Do you believe in aliens?'',  ``What is your eye color?'' and ``How tall are you?''. 

Figure \ref{fig:sb} shows an example storyboard of a user exploring the aforementioned survey dataset through \system{}. A user, Eve, starts out by looking at two attributes she is interested in: age and height (A top). She wants to see if there is a correlation between the two and creates a third visualization where she plots them against each other (A bottom). Just from visual inspection age and height do not seem to be correlated. However, Eve wants to be sure and explicitly creates a hypothesis test. She uses a multitouch gesture (dragging the two visualization close to each other) and the system automatically picks and computes an appropriate test (a correlation test in this case) (B). This confirms Eve's intuition that the two attributes are not correlated. Eve continues to look at the answers to the question ``Do you know what SQL is?''. It looks like a bit more than half the people answered this question with yes (C left). Our user wants to find out what other attributes are good predictors if someone knows SQL. Her first hunch is to look at age. She creates a query that allows her to filter the SQL attribute by people who are over 50 years old. For this age group the y-axis ordering of the two bars switched: less people in this age group know of SQL (C right). Visually it seems that this is quite a big effect, however the system automatically executed a hypothesis test for this comparison that tells Eve this is in fact not statistically significant (C, red block on the right-hand side). Eve goes on to do a similar query. This time checking if people who know who Mike Stonebraker is have a higher chance of knowing SQL. In this case, the hypothesis test automatically computed by \system{} reinforces what Eve sees: this is a significant effect. 

Eve realizes this manual exploration is becoming fairly laborious and decides to continue by using the automatic visualization recommender that \system{} provides. By tapping on the SQL attribute visualization she gets a handle to invoke a search for recommendation visualization. The handle shows that she has 70\% of exploration budget left (E). The visualisation recommendation system works through touch gestures. Dragging away from the handle invokes it, whereby the angle and the length of the drag-path determine the amount of exploration budget that should be spent and if the system should look for similar or dissimilar visualization (+ and - signs on the handle) (F and G). Eve chooses to spend 20\% of her budget and to look for dissimilar visualizations (G). The system progressively computes and ranks recommendations until the allotted budget is used up and presents thumbnails of results (H). Eve can press-and-hold on thumbnails to preview the result. Overlaid in purple she sees that people who know programming have a higher chance of also knowing SQL (I). She wants to see more detail about that relation and drags the thumbnail out which in turn creates a filter chain similar to the one she created manually before. 

All hypothesis test results are automatically tracked (and subjected to multiple hypothesis correction) by the system and displayed in a scrollable list through which Eve can, at any time, inspect and obtain additional information about any hypothesis (Figure \ref{fig:ui}). 


\section{Conclusion}
\label{sec:conclusion}
Visual data exploration and recommendation systems allow users to examine large number of observations either manually or automatically. 
If not treated as hypotheses and correctly controlled, visually significant results might be mistaken as statistically significant. 
Recent advance in multiple comparison problem and control procedures adds the desirable property of interactivity to false discovery control for data exploration~\cite{zhao2016controlling}.  
\system{} implements these interactive procedures and exposes them through a pen-\&-touch UI that allows for explicit and implicit formulation of hypotheses and user-driven steering of visualization recommendation.

%\section{Acknowledgments}
%This research is funded in part by the Intel Science and Technology Center for Big Data, the NSF CAREER Award IIS-1453171, the Air Force YIP AWARD FA9550-15-1-0144, NSF IIS-1514491, and gifts from SAP, Oracle, Google, Mellanox, and Amazon.

%\clearpage
\balance
\begin{scriptsize}
\bibliographystyle{abbrv}
\bibliography{bib}
\end{scriptsize}

\end{document}