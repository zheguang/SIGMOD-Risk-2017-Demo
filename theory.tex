\section{Controlling False Discoveries}
\label{sec:theory}

The theory of controlling false discovery in interactive data exploration is introduced in~\cite{zhao2016controlling}. 
Data exploration is modeled as a growing sequence of observations.  Three different exploration settings are studied, namely, the \textit{targeted exploration}, the \textit{free-form exploration}, and the \textit{uniform exploration}~\cite{zhao2016controlling}.  In targeted exploration, the user focuses on a predefined set of questions around narrow topic, such as ``understanding what affects the salary distribution.''  The earlier discoveries are thus used to build up the subsequent ones.  Many data-driven scientific studies fall into this category.  On the other hand, the user may not start with a focus, but may be interested in developing a focus as more data are explored.  Such process is prevalent in interactive data exploration.  Finally, when data exploration is less structured, such as in the case of recommendation engines, significant insights can be modeled as uniformly distributed in the process.

A procedure for false discovery control evaluates each observation by a corresponding hypothesis test, which outputs a \pval{}. A control procedure such as $\beta$-Farsighted~\cite{zhao2016controlling} starts with a predefined amount of exploration \textit{budget}, and invests a fraction of the budget as the significance level for each test.  If the observation is deemed significant, or equivalently, the test is rejected, then a fraction of the investment is returned to the budget. The procedure halts when either there are no more tests or the budget reduces to zero.  The guarantee for the entire data exploration is that the \textit{marginalized false discovery rate} (mFDR), namely, the ratio between the expected number of false discoveries and that of all discoveries, is no greater than the fraction set as the initial exploration budget.

Contextual information is useful for the quality of false discovery control. If the number of observations the exploration is unknown or unbounded, the best strategy to use is $\beta$-Farsighted~\cite{zhao2016controlling}, which always conserves a proportion of the current exploration budget.  Otherwise if the expected exploration length can be approximated, then $\epsilon$-Hybrid~\cite{zhao2016controlling} offers similar power but lower false discovery rate.  These procedures are most efficient in terms of power and error rates for the aforementioned exploration settings~\cite{zhao2016controlling}.

%If the exploration budget reaches zero, the per-observation investment can be reduced a posteriori for the past observations.  Because the investment corresponds to the per-test significance level, some previously significant observation may become insignificant, but on the other hand each investment lost is also reduced.  If the reduction of investment regains eventual exploration budget, then the exploration can continue.  Such reduction policy is analogous to the Bonferroni procedure~\cite{bonferroni1936teoria}, where the significance level for each test reduces as more observations are made.

%The previously known control procedures such as Bonferroni~\cite{bonferroni1936teoria}, FDR~\cite{BenjaminiH95} and Sequential FDR~\cite{g2016sequential} are not interactive, because the decisions are only finalized when all observations are made.  The main advantage of the \ainv{} procedures proposed in~\cite{zhao2016controlling} is interactivity , where observations can be made dynamically.